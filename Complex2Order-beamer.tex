% !TEX encoding = UTF-8 Unicode
% !TEX TS-program = XeLaTeX
% !TEX spellcheck = English
% !TEX pdfSinglePage



\documentclass[14pt]{beamer}
	\makeatletter
	\overfullrule1em
	\beamertemplatenavigationsymbolsempty
	\setbeamercovered{transparent}
	\setbeamersize{text margin left=3mm,text margin right=3mm}
	\def\CMH{\gdef\beamer@currentmode{handout}}
	\def\CMB{\gdef\beamer@currentmode{beamer}}
	
	\catcode`激13 \def激#1{\lccode`~`#1\lowercase{\catcode`#113\def~}}
	激色#1!#2 {\definecolor{#1}{HTML}{#2}}% www.uillinois.edu/OUR/brand/color_palettes
	色UIB!13294b				色2728 C!0455A4			色2738 C!1F4096			% blues
	色427!E8E9EA				色Cool Gray 6!A5A8AA		色Cool Gray 1!5E6669		% neutrals
	色UCO!E84A27				色UIC red!D50032			色UIS blue!003366		% accents
	色Teal!0d605e			色Gray-blue!6fafc7		色Citron!bfd46d			% accents
	色Dark yellow!ffd125		色Salmon!ee5e5e			色Periwinkle!4f6898		% accents
	\setbeamercolor{normal text}{bg=UIB,fg=white}
	\setbeamercolor{structure}{fg=UCO}
	\setbeamercolor{alerted text}{fg=Dark yellow}
	\setbeamercolor{example text}{fg=Citron}

\usepackage{xurl}
	\hypersetup{pdfsubject=Information Theory (cs.IT)}
	\hypersetup{colorlinks,linkcolor=UIC red,citecolor=Citron,urlcolor=Gray-blue}

\usepackage{mathtools}

\usepackage[warnings-off={mathtools-colon,mathtools-overbracket}]{unicode-math}
	\setmainfont{Times}
	\setsansfont{HelveticaNeue-Light}
	\setmonofont{Menlo}
	\setmathfont[sans-style=literal]{texgyrepagella-math.otf}
	
	\DeclareMathOperator*\argmax{argmax}
	\DeclareMathOperator*\mybest{do-my-best}
	\DeclareMathOperator*\wt{wt}
	\def\P{P_\mathrm e}
	
	\def\({\bigl(}	\def\){\bigr)}	激({\Bigl(}		激){\Bigr)}		
	激[{\bigl[}		激]{\bigr]}		激「{\Bigl[}		激」{\Bigr]}		
	激{{\bigl\{}	激}{\bigr\}}	激『{\Bigl\{}	激』{\Bigr\}}	
	激【{\lvert}		激】{\rvert}	
	
	激㏒{\log}		激㏑{\ln}		
	激ˆ{\hat}		激¯{\bar}		
	激|{\mid}		激:{\colon}		激;{\mathrel;\nobreak}		
	激÷{\frac}		激√{\sqrt}		激¬{\limits}		
	激†#1†{{\text{#1}}}				
	激⋆{\raisebox{1ex}{$\star$}}
	
	\def\K#1{^{(#1)}}
	\def\W#1{W\K{#1}}
	\def\WW#1#2{(\W{#1})\K{#2}}
	\def\WWW#1#2#3{\(\WW{#1}{#2}\)\K{#3}}
	\def\V#1{V\K{#1}}
	\def\VV#1#2{(\V{#1})\K{#2}}
	\def\VVV#1#2#3{\(\VV{#1}{#2}\)\K{#3}}
	\def\G#1{G\K{#1}}
	\def\GG#1#2{\G{#1#2}}
	
	\def\pp{\pause\par}
	\advance\parskip\fill

\usepackage{tikz,tikz-cd}
	\pgfmathsetseed{24315}\let\PMP\pgfmathparse\def\PMR{\pgfmathresult}
	\let\PMS\pgfmathsetmacro\let\PMT\pgfmathtruncatemacro\let\PMD\pgfmathdeclarefunction
	% https://tex.stackexchange.com/q/420034/
	\PMD*{axis_height}0{\begingroup\pgfmathreturn.25em\endgroup}
	\PMD*{rule_thickness}0{\begingroup\pgfmathreturn.06em\endgroup}
	\tikzset{
		every picture/.style={cap=round,join=round,line width=rule_thickness},
		% https://tex.stackexchange.com/q/146908/
		alt/.code args={<#1>#2#3}{\alt<#1>{\pgfkeysalso{#2}}{\pgfkeysalso{#3}}},
		uncover/.style={alt=<#1>{}{opacity=.15}},
		dynamic opacity/.style={opacity=(7/8)^\d}
	}
	\def\recursivetree#1#2{
		\edef\d{#1}\edef\z{#2}
		\fill[dynamic opacity](\d,\z/32)circle(2pt);
		\ifnum\d<\depth
		\ifdim\z pt>\threshold pt
		\ifdim\z pt<\holdthres pt
			\edef\dplusplus{\the\numexpr\d+1}
			{
				\PMS\zup{\z*(256-\z)/128}
				\draw[dynamic opacity](\d,\z/32)--(\dplusplus,\zup/32);
				\recursivetree\dplusplus\zup
			}
			{
				\PMS\zdown{\z^2/128}
				\draw[dynamic opacity](\d,\z/32)--(\dplusplus,\zdown/32);
				\recursivetree\dplusplus\zdown
			}
		\fi
		\fi
		\fi
	}

\usepackage{pgfplotstable,booktabs,colortbl}
	\pgfplotsset{compat/.cd,show suggested version=false,=1.17}
	\pgfplotstableset{
		every head row/.style={before row=\toprule,after row=\midrule},
		every last row/.style={after row=\bottomrule},string type,
	}
	\def\arraystretch{1.5}

\title[Complexity \& 2-Term of Achieving Capacity]
	{Complexity and the 2nd-Order Term\\of Capacity-Achieving Codes}
\author[H-P\ Wang]{Hsin-Po WANG}
\institute{Department of Mathematics, University of Illinois at Urbana--Champaign}
\date[2020-10]{2020-10-05 CSL SINE Group Seminar}

\begin{document}

\frame\maketitle

\makeatletter\defbeamertemplate*{sidebar right}{thinbold}{
	\tikz[remember picture,overlay,x=3mm,y=\paperheight]{\footnotesize
		\PMS\olfrac{\insertoverlaynumber/(\insertframeendpage+1-\insertframestartpage)}
		\path[save path=\stare,yscale={1/max(\insertmainframenumber-1,1)}]
			(-1,0)-|(0,1-\insertframenumber)-|+(-\olfrac,1)-|cycle;
		\tikzset{banner/.pic={\node at(-.55,-.5)[rotate=-90]
			{\beamer@shorttitle~\beamer@shortdate~\beamer@shortauthor};}}
		\fill[use path=\stare,Cool Gray 1]pic[Cool Gray 6]{banner};
		\clip[use path=\stare]pic[UIB]{banner};
	}
}

%\CMH

\frame{{Noisy channel}
	The sender inputs  $X₁³²=\texttt{11001001 00001111 11011010 10100010}$.
	\pp
	The channel output $Y₁³²=\texttt{1--01-01 ----1--- -101---0 --0--0-0}$.
}

\frame{{Noisy channel}
	Sender inputs $X₁³²∈𝔽_q³²$, where $𝔽_q$ is input alphabet.	\\
	We may assume $𝔽_q$ is a finite field [new idea].
	
	Channel outputs $Y₁³²$ according to stochastic matrix
	$ℙ\{Y_i=y|X_i=x\}=W(y|x)$ independently for each $i$.
}

\frame{{Noisy channel coding}
	The sender inputs $X₁³²∈ℬ⊊𝔽_q^{32}$.	\\
	$ℬ$ is a block code (codebook) of block length $N=32$.
	\pp
	The channel output	$Y₁³²$ according to $W(y|x)$.
	\pp
	The receiver maximize the a posterior probability
	$ˆX₁³²={\alt<+(1)>{\color{alerted text.fg}\mybest}\argmax
		¬_{x₁³²∈ℬ}}ℙ\{X₁³²=x₁³²|Y₁³²\}$.
}

\frame{{Noisy channel coding theorem}
	Channel capacity $C≔\sup¬_{\!X∼Q\!}I(X;Y)$ (mutual information).	\\
	Block length is $N$.	\\
	Error probability is $\P≔ℙ\{ˆX₁^N≠X₁^N\}$.	\\
	Code rate is $R≔㏒【ℬ】/N㏒q$	\hfill(recall that $ℬ⊂𝔽_q^N$).	
	\pp
	[Shannon 1948]
	\emph{
		One can find block code $ℬ$	\\
		such that $\P→0$ and $R→C$ as $N→∞$.	\\
		(And $C$ is the greatest number that makes this hold.)
	}
	%%%%% add Shannon photo
}

\frame{{2nd-order term of the theorem}
	How fast do error probability $\P$ and code rate $R$
	converge to $0$ and $C$ as block length $N→∞$?	\\
	Characterize them as functions ``$\P(N)$'' and ``$R(N)$''.
	\pp
	When $R$ is fixed, $\P≈e^{-N}$, that is, $-㏒\P≈N$.	\\
	When $\P$ is fixed, $R≈C-N^{-1/2}$, that is, $(C-R)^{-2}≈N$.
	\pp
	When both $R$ and $\P$ vary, $(-㏒\P)(C-R)^{-2}≈N$.
}

\frame{{2nd-order term analysis}
	This is two-sided bound:	\\
	A code $ℬ$ exists such that $(-㏒\P)(C-R)^{-2}≈N$.	\\
	$ℬ$ does not exist such that $(-㏒\P)(C-R)^{-2}≫ N$.
	\pp
	Block length $N$ is your income;	\\
	invest error probability $\P$ or code rate $R$ or both.
}

激分#1#2{÷{#1\rule{0ex}{1.5ex}}{#2\rule[-1ex]{0ex}{0ex}}}
\pgfplotstableread[col sep=ampersand]{
	P.	&	Paradigm						&	Random variable						
	LLN	&	law of large numbers			&	¯X→μ								
	LDP	&	large deviations principle		&	ℙ\{¯X-μ>x\}≈e^{-nI(x)}				
	CLT	&	central limit theorem			&	¯X-μ∼𝒩(0,σ√n)						
	MDP	&	moderate deviations principle	&	分{-㏒ℙ\{¯X-μ>ε_nx\}}{ε_n²}≈nI(x)	
}\tableTrinity
\pgfplotstableread[col sep=ampersand]{
	P.	&	Random code			&	Polar code					
	LLN	&	(\P,R)→(0,C)		&	(\P,R)→(0,C)				
	LDP	&	\P≈e^{-N}			&	\P≈e^{-N^π}					
	CLT	&	C-R≈N^{-1/2}		&	C-R≈N^{-ρ}					
	MDP	&	分{-㏒\P}{(C-R)²}≈N	&	(\P,C-R)≈(e^{-N^π},N^{-ρ})	
}\tableCoding
\pgfplotstablecreatecol[create col/copy column from table=\tableCoding{Random code}]
	{Random code}\tableTrinity
\pgfplotstablecreatecol[create col/copy column from table=\tableCoding{Polar code}]
	{Polar code}\tableTrinity
\pgfplotstablemodifyeachcolumnelement{Paradigm}\of\tableTrinity\as\cell
	{\edef\cell{\noexpand\small\unexpanded\expandafter{\cell}}}
\pgfplotstablemodifyeachcolumnelement{Random variable}\of\tableTrinity\as\cell
	{\edef\cell{$\unexpanded\expandafter{\cell}$}}
\pgfplotstablemodifyeachcolumnelement{Random code}\of\tableTrinity\as\cell
	{\edef\cell{$\unexpanded\expandafter{\cell}$}}
\pgfplotstablemodifyeachcolumnelement{Polar code}\of\tableTrinity\as\cell
	{\edef\cell{$\unexpanded\expandafter{\cell}$}}
\frame{{2nd-order term analog}
	\centering
	\pgfplotstabletypeset[
		columns={Paradigm,Random variable},
		columns/Random variable/.style={column type={>{\onslide<2>}c}},
	]\tableTrinity
}

\frame{{2nd-order term analog}
	\centering
	\pgfplotstabletypeset[
		columns={P.,Random variable,Random code},
		columns/Random code/.style={column type={>{\onslide<2>}c}},
	]\tableTrinity
}

\frame{{However...}
	The achievability bound for random code $ℬ$	\\
	assumes exponential complexity due to $\argmax¬_{x₁³²∈ℬ}$.
	\pp
	Goal:
	Comparable performance,	\\
	but with a low-complexity decoder $\mybest¬_{x₁³²∈ℬ}$.
}

\frame{{2nd-order term goal}
	\centering
	\pgfplotstabletypeset[
		columns={P.,Random code,Polar code},
		columns/Polar code/.style={
			column type={>{\onslide<2>}c},column name=Low-complexity code},
	]\tableTrinity
	
	\hfill\onslide<2>($0<π,ρ$ and $π+2ρ<1)$
}

\pgfplotstableread[col sep=ampersand]{
	P.		&	binary	&	prime-ary	&	finite		&	asymmetric	
	LDP⋆	&	k		&	k			&	k			&	k			
	MDP⋆	&	k		&	k			&	w			&	w			
	LDP		&	k		&	w			&	w			&	w			
	CLT		&	k		&	w			&	w			&	w			
}\tableKnown
\def\assigncontent#1{\pgfkeyssetvalue{/pgfplots/table/@cell content}{#1}}
\def\decodecontent#1#2\relax{
	\if\pgfplotstablecol0	\assigncontent{#1#2}
	\else\if#1w				\assigncontent{???}
	\else					\assigncontent{known}
	\fi\fi
}
\frame{{Polar coding}
	[Arıkan 2009] invented polar coding.
	It produces practical codes with provable bounds on $\P$ and $R$.
	\pp
	\centering\pgfplotstabletypeset[
		assign cell content/.code={\decodecontent####1\relax}
	]\tableKnown
}


\frame{{Polar coding road map}
	\textcolor{example text.fg}{Channel transformation} manipulates channels.
	
	\textcolor{example text.fg}{Channel tree} is the result of recursive transformation.
	
	\textcolor{example text.fg}{Channel parameter} measuress the reliability of channels.
	
	\textcolor{example text.fg}{Channel process} is syntax candy (very useful).
	
	\textcolor{example text.fg}{Channel polarization} is a phenomenon.
}

\frame{{Channel transformation}
	Channel $W=(X|Y)$; input $X$; output $Y$.	\\
	Make i.i.d.\ copies $(X₁|Y₁)$ and $(X₂|Y₂)$.
	\pp
	$\W1≔(X₁-X₂|Y₁²)$;	\\
	$\W2≔(X₂|(X₁-X₂)Y₁²)$	\hfill(juxtaposition is tupling).
}

\frame{{Channel transformation (other kernel)}
	$U₁²$ two free variables; $G$ a $2×2$ matrix (called kernel);	\\
	$X₁²≔U₁²·G$; channels generate $Y₁²$.
	\pp
	$\W1≔(U₁|Y₁²)$;	\\
	$\W2≔(U₂|U₁Y₁²)$	\hfill(juxtaposition is tupling).
}

\frame{{Channel transformation (larger kernel)}
	$U₁^ℓ$ this many free variables; $G$ an $ℓ×ℓ$ kernel matrix;	\\
	$X₁^ℓ≔U₁^ℓ·G$; channels generate $Y₁^ℓ$.
	\pp
	$\W1≔(U₁|Y₁^ℓ)$;	\\
	$\W2≔(U₂|U₁Y₁^ℓ)$;	\\
	$\W3≔(U₃|U₁²Y₁^ℓ)$;	\\
	$\phantom{\W{ℓ-1}}⋮$	\\
	$\W{ℓ-1}≔(U_ℓ|U₁^{ℓ-2}Y₁^ℓ)$;	\\
	$\Wℓ≔(U_ℓ|U₁^{ℓ-1}Y₁^ℓ)$.
	\pause
	\tikz[x=1em,y=-1em,overlay,shift={(5,-6)}]{
		\draw
			(0,.5)rectangle node(G){$·G$}(3,5.5)
			foreach\k in{1,...,5}{
				(-2,\k)node(U\k){$U_\k$}(U\k)-|(0,3)
				(5,\k)node(X\k){$X_\k$}(3,3)|-(X\k)
				(8,\k)node(Y\k){$Y_\k$}(X\k)--(Y\k)
			}
		;
	}
}

\frame{{Channel tree}
	Channel $W$ grows $\W1,\W2,…,\Wℓ$.
	\pp
	For each $i$, channel $\W i$ grows $\WW i1,…,\WW iℓ$.
	\pp
	For each $j$, channel $\WW ij$ grows $\WWW ij1,…$.
	\pp
	\centering
	\tikz[
		grow=right,
		level/.style={
			level distance=2em*####1,sibling distance=8em/2^####1,nodes={scale=7/8}
		}
	]{
		\node{$W$}
		child foreach\i in{1,2}{
			node{$\W\i$}
			child foreach\j in{1,2}{
				node{$\WW\i\j$}
				child foreach\k in{1,2}{
					node{$\WWW\i\j\k$}
				}
			}
		}
	}
}

\frame{{Dynamic kernel [new idea*]}
	Channel $W$ grows $\W1,\W2,…,\Wℓ$ using $G$.
	\pp
	Channel $\W i$ grows $\WW i1,…,\WW iℓ$ using $\G i$.
	\pp
	Channel $\WW ij$ grows $\WWW ij1,…$ using $\GG ij$.
	\pp
	\centering
	\tikz[
		grow=right,
		level/.style={
			level distance=2em*####1,
			sibling distance=8em/2^####1,
			nodes={scale=7/8}
		}
	]{
		\node{$W$}
		child foreach\i in{1,2}{
			node{$\W\i$}
			child foreach\j in{1,2}{
				node{$\WW\i\j$}
				child foreach\k in{1,2}{
					node{$\WWW\i\j\k$}
				}
			}
		}
	}
}

%	$W$ grows $\W1$.
%	$\W1$ grows $\WW11$ and $\WW12$;
%	$\W2$ grows $\WW21$ and $\WW22$.
%	$\WW11$ grows $\WWW111$ and $\WWW112$;
%	$\WW12$ grows $\WWW121$ and $\WWW122$;
%	$\WW21$ grows $\WWW211$ and $\WWW212$;
%	$\WW22$ grows $\WWW221$ and $\WWW222$.

\frame{{Channel parameter ($ℓ=2$ and $n=3$)}
	Block length $N=ℓ^n=2³=8$.
	\pp
	Select indices $ℐ≔\{212,221,222\}∈\{1,2\}³$.	\\
	Code rate $R=【ℐ】/N=3/8$ (nontrivial).
	\pp
	Error probability $\P≤∑¬_{ijk∈ℐ}H(\WWW ijk)$ (nontrivial);	\\
	$H(X|Y)$ is conditional entropy (base to be specify).
}

\frame{{It suffices to understand}
	$H(W),H(\W i),H\(\WW ij\),H(\WWW ijk),…$.
	\pp
	Block length $N$ will be $ℓ^†where we stop†$.
	\pp
	Code rate $R$ will be the fraction of small $H$-values.
	\pp
	Error probability $\P$ will be $∑¬_†those†$ small $H$-values.
}

\frame{{Channel process (syntax candy)}
	$𝘞₀≔W$.	\\
	$𝘞_{n+1}≔𝘞_n^{(𝘒_{n+1})}$, where $𝘒_{n+1}∈\{1,2,…,ℓ\}$ i.i.d.\ uniform.
	\pp
	$𝘏_n≔H(𝘞_n)$.
	\pp
	Decide depth $n$, then block length $N=ℓ^n$.	\\
	Decide threshold $θ$, then code rate $R=ℙ\{𝘏_n<θ\}$.	\\
	Error probability $\P<∑†small †𝘏_n<∑θ=RNθ≤Nθ$.
}

\frame{{Channel polarization}
	$𝘏_n≔H(𝘞_n)$ is a martingale. (Invoke the Doob's.)
	$𝘏_n→𝘏_∞$ a.e.\ as $n→∞$; it turns out $𝘏_∞∈\{0,1\}$.
	\pp
	\tikz{
		\def\depth{6}\def\threshold{0}\def\holdthres{128}
		\recursivetree0{50}
		\pause
		\draw[alerted text.fg](\depth,1/4)--+(1,0)node[right]{threshold $θ$};
	}
%	\pp
%	A great question: characterize the pace of convergence.
}

\frame{{It suffices to understand}
	\par
	{\LARGE$ℙ\{𝘏_n<†threshold†\}>C-†gap†$.\par}
	\pp
	Goal: $ℙ{𝘏_n<e^{-ℓ^{πn}}}>C-ℓ^{-ρn}$, where $π+2ρ<1$.	\\
	Then $N=ℓ^n$ and $\P<Ne^{-N^π}$ and $R>C-N^{-ρ}$ .
}

\frame{{Proof outline}
	\textcolor{example text.fg}{Local LDP behavior:}
	$Z(\W k)≤ℓe^{qZ(W)ℓ}(qZ(W))^{⌈k^2/3ℓ⌉}$.	\\
	(Never heard Bhattacharyya parameter? $Z≔H$.)
	
	\textcolor{example text.fg}{Local CLT behavior:}
	$∑¬_{k=1}^ℓf(H(\W k))<4ℓ^{1/2+α}$,	\\%%%%%%% replace h by f
	where $α=㏒㏒ℓ/㏒ℓ$ and $f(z)≔\min(z,1-z)^α$.
	
	\textcolor{example text.fg}{Global MDP behavior:}
	$ℙ{𝘏_n<e^{-ℓ^{πn}}}>C-ℓ^{-ρn}$,
	where $π+2ρ<1$, given local LDP and local CLT behaviors.
}

\frame{{Local LDP behavior 1/3}
	Want to prove $Z(\W k)≤ℓe^{qZ(W)ℓ}(qZ(W))^{⌈k^2/3ℓ⌉}$.	\\
	Let $z≔Z(W)$; want $Z(\W k)≤ℓe^{qzℓ}(qz)^{⌈k^2/3ℓ⌉}$.
	\pp
	Lemma: $Z(\W k)≤∑¬_{u_{k+1}^ℓ∈𝔽_q^{ℓ-k}}z^{\wt(0₁^{k-1}1_ku_{k+1}^ℓ·G)}$;	\\
	RHS is weight enumerator of a coset code.
	
	$\W1≔(U₁|Y₁^ℓ)$;	\\
	$\W2≔(U₂|U₁Y₁^ℓ)$;	\\
	$\phantom{\W2}⋮$	\\
	$\Wℓ≔(U_ℓ|U₁^{ℓ-1}Y₁^ℓ)$.
	\tikz[x=1em,y=-1em,overlay,shift={(5,-5)}]{
		\draw
			(0,.5)rectangle node(G){$·G$}(3,5.5)
			foreach\k in{1,...,5}{
				(-2,\k)node(U\k){$U_\k$}(U\k)-|(0,3)
				(5,\k)node(X\k){$X_\k$}(3,3)|-(X\k)
				(8,\k)node(Y\k){$Y_\k$}(X\k)--(Y\k)
			}
		;
	}
}

\frame{{Local LDP behavior 2/3}
	Want $∑¬_{u_{k+1}^ℓ}z^{\wt(0₁^{k-1}1_ku_{k+1}^ℓ·G)}
		≤ℓe^{qzℓ}(qz)^{⌈k^2/3ℓ⌉}$ for some $G$.
	% In fact, RHS is $ℓ(1+(q-1)Z(W))^{ℓ-⌈k^2/3ℓ⌉}((q-1)Z(W))^{⌈k^2/3ℓ⌉}$.	\\
	\pp
	$G$ random; $𝔼†LHS†=q^{-k}(1+(q-1)z)^ℓ≤q^{-k}(1+qz)^ℓ$.
	\pp
	Compare $(qz)^w$-coefficients:
	$q^{-k}\binomℓw$ vs $ℓ÷{ℓ^{w-⌈k^2/3ℓ⌉}}{(w-⌈k^2/3ℓ⌉)!}$.
	\pp
	Simplify: $2^{-k}\binomℓ{⌈k^2/3ℓ⌉}\binom{ℓ-⌈k^2/3ℓ⌉}{w-⌈k^2/3ℓ⌉}$
	vs $ℓ\binomℓ{w-⌈k^2/3ℓ⌉}$.
}

\PMD{hleft}1{%
	\PMS\X{2^-#1}%
	\PMS\Y{1-\X}%
	\PMP{\X*#1-\Y*log2(\Y)}%
}
\PMD{hmiddle}1{%
	\PMS\Y{1-#1}%
	\PMP{-#1*log2(#1)-\Y*log2(\Y)}%
}
\frame{{Local LDP behavior 3/3}
	Boils down to $2^{-k}\binomℓ{⌈k^2/3ℓ⌉}$ vs $ℓ$; ignore/cancel $⌈⌉$ and $ℓ$.
	\pp
	$\binomℓd$ is about $2^{ℓh_2(d/ℓ)}$ for $d=Θ(ℓ)$. (Large deviations!)	\\
	Hence $k$ vs $h_2(k^2/3ℓ^2)$, which becomes $√{3x}$ vs $h_2(x)$.
	\pp
	\tikz[scale=3]{
		\draw[domain=2:7]
			plot(2^-\x,{hleft(\x)})--(0,0)plot(1-2^-\x,{hleft(\x)})--(1,0);
		\draw[domain=1/4:3/4]plot(\x,{hmiddle(\x)});
		\draw plot[domain=0:1](\x^2/e,\x);
	}
	\hfill zoom $→$\hfill
	\tikz[scale=9]{
		\draw[domain=1/6:1/2]plot(\x,{hmiddle(\x)});
		\draw plot[domain=2/3:1](\x^2/e,\x);
	}
}

\frame{{Local CLT behavior 1/4}
	Want to prove $∑¬_{k=1}^ℓf(H(\W k))<4ℓ^{1/2+α}$.
	\pp
	Break into segments
	$\begin{cases}
		∑_{k=H(W)+ℓ^{-1/2+α}}^ℓ<ℓ^{1/2+α},	\\[2ex]
		∑_{k=H(W)-ℓ^{-1/2+α}}^{H(W)+ℓ^{-1/2+α}}<2ℓ^{1/2+α}, \\[2ex]
		∑_{k=1}^{H(W)-ℓ^{-1/2+α}}<ℓ^{1/2+α}.
	\end{cases}$
	\pp
	\tikz[overlay,shift={(1,-1)}]{
		\draw(0,0)[domain=0:2.381101578]plot(\x^3/9,\x)plot(3-\x^3/9,\x);
		\fill foreach\y in{.1,.2,.3,.5,.8,1.3,2.1}{(\y^3/9,\y)circle(2pt)};
		\fill foreach\y in{.1,.2,.4,.8,1.6}{(3-\y^3/9,\y)circle(2pt)};
	}
}

\frame{{Local CLT behavior 2/4}
	Want to show $∑¬_{k=H(W)+ℓ^{-1/2+α}}^ℓf(H(\W k))<ℓ^{1/2+α}$.
	\pp
	Jensen LHS: $(ℓ-m)f(÷1{ℓ-m}∑¬_{k=m+1}^ℓH(\W k))<ℓ^{1/2+α}$,
	where $m=H(W)+ℓ^{-1/2+α}-1$.
	\pp
	$\phantom{\W{ℓ-2}}⋮$	\\
	$\W{ℓ-2}≔(U_{ℓ-2}|U₁^{ℓ-3}Y₁^ℓ)$,	\\
	$\W{ℓ-1}≔(U_{ℓ-1}|U₁^{ℓ-2}Y₁^ℓ)$,	\\
	$\Wℓ≔(U_ℓ|U₁^{ℓ-1}Y₁^ℓ)$.
	\hfill$\smash{∑¬_{k=m+1}^ℓ}=H(U_{m+1}^ℓ|U₁^mY₁^ℓ)$.
}

\frame{{Local CLT behavior 3/4}
	$H(U_{m+1}^ℓ|U₁^mY₁^ℓ)$ is what? ($m=H(W)+ℓ^{-1/2+α}-1$)
	\pp
	The conditional entropy of	\\
	noisy channel coding.
	\tikz[x=1em,y=-1em,overlay,shift={(5,-3)}]{
		\draw
			(0,.5)rectangle node(G){$·G$}(3,5.5)
			foreach\k in{1,...,5}{
				(-2,\k)node(U\k)[opacity={\k<3?0:1}]{$U_\k$}(U\k)-|(0,3)
				(5,\k)node(X\k){$X_\k$}(3,3)|-(X\k)
				(8,\k)node(Y\k){$Y_\k$}(X\k)--(Y\k)
			}
		;
	}
	\pp
	Gallager has good bounds.
}

\frame{{Local CLT behavior 4/4}
	The other segment: $∑¬_{k=1}^{H(W)-ℓ^{-1/2+α}}f(H(\W k))<4ℓ^{1/2+α}$.	\\
	Jensen inequality: $mf(÷1m∑¬_{k=1}^{m+1}H(\W k))<4ℓ^{1/2+α}$.
	Chain rule: $H(U₁^m|Y₁^ℓ)$, what is this? \uncover<+(1)->{Guess?}
	
	\visible<+(1)->{
		wiretap channel;	\\
		Hayashi has	\\
		good bounds	\\
		{}[new idea].
		\tikz[x=1em,y=-1em,overlay,shift={(10,-5)}]{
			\def\comment{\llap{\ifnum\k<3 message \else obscure \fi}}
			\draw
				(0,.5)rectangle node(G){$·G$}(3,5.5)
				foreach\k in{1,...,5}{
					(-2,\k)node(U\k){\comment$U_\k$}(U\k)-|(0,3)
					(5,\k)node(X\k){$X_\k$}(3,3)|-(X\k)
					(8,\k)node(Y\k){$Y_\k$}(X\k)--(Y\k)
				}
			;
		}
	}
}

\frame{{A calculus machinery [new idea]}
	local LDP behavior: $Z(\W k)≤ℓe^{qZ(W)ℓ}(qZ(W))^{⌈k^2/3ℓ⌉}$.	\\
	Local CLT behavior: $∑¬_{k=1}^ℓf(H(\W k))<4ℓ^{1/2+α}$.
	\pp
	eigen: $𝔼[f(𝘏_{n+1})|𝘏₀,…,𝘏_n]≤ℓ^{-1/2+3α}f(𝘏_n)$.
	\pp
	en23: $ℙ{𝘡_n<e^{-n^{2/3}}}>C-ℓ^{(-1/2+4α)n}$.
	\pp
	een13: $ℙ『𝘡_n<\exp\(-e^{n^{1/3}}\)』>C-ℓ^{(-1/2+4α)n}$.
	\pp
	elpin: $ℙ{𝘡_n<e^{-ℓ^{πn}}}>C-ℓ^{-ρn}$.
}

\frame{{Summary so far}
	For all $π+2ρ<1$, there exist codes with	\\
	error probability $\P<e^{-N^π}$	and code rate $R>C-N^{-ρ}$.
	\pp
	When only $2×2$ kernels are allowed, at least $π,ρ>0$.
	\pp
	It happens that they have complexity $O(㏒N)$ per bit.
	\pp
	Can we reduce the complexity further	\\
	(at the expense of worse performance etc)?
}

\frame{{Pruning}
	The bottom channel is good enough	\\
	before we reach our favorite $n$.
	
	\tikz{
		\path(0,0)circle(2pt)(0,4)circle(2pt);
		\def\depth{2}\def\threshold{8}\def\holdthres{128}
		\only<3>{\def\depth{3}}
		\only<4>{\def\depth{4}}
		\only<5>{\def\depth{5}}
		\only<6>{\def\depth{6}}
		\only<7>{\def\depth{7}}
		\recursivetree0{50}
		\draw[alerted text.fg](2,1/4)--+(\depth-1,0)node[right]{threshold $θ$};
	}
	\pp
	Why do we apply transform any further? (Ans: don't!)
}

\frame{{Pruning}
	The top channel is too bad.
	Do we expect	\\
	any of its descendants to be good enough?
	
	\tikz{
		\path(0,0)circle(2pt)(0,4)circle(2pt);
		\def\depth{3}\def\threshold{8}\def\holdthres{120}
		\only<3>{\def\depth{4}}
		\only<4>{\def\depth{5}}
		\only<5>{\def\depth{6}}
		\only<6>{\def\depth{7}}
		\only<7>{\def\depth{8}}
		\recursivetree0{50}
		\draw[example text.fg](3,4-1/4)--+(\depth-2.7,0)node[right]{threshold $1-θ$};
		\draw[alerted text.fg](2,1/4)--+(\depth-1,0)node[right]{threshold $θ$};
	}
	\pp
	We don't.
}

\frame{{Stopping time}
	Channel $𝘏_i$ needs transformation if $θ<𝘏_i<1-θ$.
	\pp
	Set $θ=N^{-10}$; assume $i>O(㏒㏒N)$, then $e^{-2^{πi}}<θ$.
	\pp
	Then $ℙ\{𝘏_i≤θ\}>ℙ{𝘏_i≤e^{-2^{πi}}}≥C-ℓ^{-ρi}$
	and $ℙ\{1-θ≤𝘏_i\}>ℙ{1-e^{-2^{πi}}≤H_i}≥1-C-ℓ^{-ρi}$
	\pp
	That is, $ℙ\{θ<𝘏_i<1-θ\}≤2ℓ^{-ρi}$.
}

\frame{{Geometric complexity}
	Complexity $=$ \#transformations $=∑¬_{i=0}^nℙ\{θ<𝘏_i<1-θ\}$.
	\pp
	$∑_{i=O(㏒㏒N)}^nℙ\{θ<𝘏_i<1-θ\}≤∑2ℓ^{-ρi}=O(1)$;	\\
	$∑_{i=0}^{O(㏒㏒N)}ℙ\{θ<𝘏_i<1-θ\}≤∑¬1=O(㏒㏒N)$.
	\pp
	Complexity is $O(㏒㏒N)$ per bit.
}

\frame{{Summary}
	\pp
	There exist codes with complexity $O(㏒㏒N)$ per bit,	\\
	error probability $\P<N^{-9}$, and code rate $R=C-N^{-ρ}$.
	\pp
	(Earlier) we have codes with complexity $O(㏒N)$ per bit,	\\
	error probability $\P<e^{-N^π}$, and code rate $R>C-N^{-ρ}$.
	\pp
	Are there codes in between?
}

\frame{{Summary}
	Log-log code taken from (with Duursma)	\\
	Log-logarithmic Time Pruned Polar Coding	\\
	\url{https://arxiv.org/abs/1905.13340}.
	
	MDP code taken from (with Duursma)	\\
	Polar Codes' Simplicity, Random Codes' Durability	\\
	\url{https://arxiv.org/abs/1912.08995}.
}

\frame{{Question?}
	\inserttitlegraphic
	
	Predefined questions:	\\
	Why input alphabet is finite field? What advantage?	\\
	Definition of Bhattacharyya parameter?	\\
	References for XYZ?	\\
	What channels? Your contribution over others?	\\
	Future plan?
}

\def\appendixname{Appendix}
\appendix

\pgfplotstableread[col sep=ampersand]{
	Code			&	Error			&	Gap		&	Complex		&	Channel		
	random			&	e^{-N^π}		&	N^{-ρ}	&	\exp(N)		&	DMC			
	RM				&	→0				&	→0		&	O(N^2)		&	BEC			
	LDPC			&	→0				&	→0		&	???			&	S.~BDMC		
	RA family		&	→0				&	→0		&	O(1)		&	BEC			
	{}[W. polar]	&	e^{-N^π}		&	N^{-ρ}	&	O(㏒N)		&	DMC			
	old prune		&	e^{-N^{1/2}}	&	O(1)	&	Θ(㏒N)		&	S.~BDMC		
	{}[W. prune]	&	N^{-9}			&	N^{-ρ}	&	O(㏒㏒N)		&	DMC			
}\tableComplex
\pgfplotstablemodifyeachcolumnelement{Error}\of\tableComplex\as\cell
	{\edef\cell{$\unexpanded\expandafter{\cell}$}}
\pgfplotstablemodifyeachcolumnelement{Gap}\of\tableComplex\as\cell
	{\edef\cell{$\unexpanded\expandafter{\cell}$}}
\pgfplotstablemodifyeachcolumnelement{Complex}\of\tableComplex\as\cell
	{\edef\cell{$\unexpanded\expandafter{\cell}$}}
\frame{
	\centering\pgfplotstabletypeset\tableComplex
}

\pgfplotstableread[col sep=ampersand]{
	P.		&	binary			&	prime-ary	&	finite		&binary2	&finite2	
	LLN		&	Arikan09		&	STA09i		&	STA09i		&SRDR12		&w			
	LDP⋆	&	AT09			&	MT14		&	Sasoglu11	&HY13		&w			
	CLT⋆	&	KMTU10,MHU16	&	BGNRS18		&	w			&w			&w			
	MDP⋆	&	GX15,MHU16		&	BGS18		&	w			&w			&w			
	LDP		&	KSU10,HMTU13	&	w			&	w			&w			&w			
	CLT		&	FHMV18,GRY20	&	w			&	w			&w			&w			
	MDP		&	w				&	w			&	w			&w			&w			
}\tableRefarray
\def\assigncontent#1{\pgfkeyssetvalue{/pgfplots/table/@cell content}{#1}}
\def\decodecontent#1#2\relax{
	\if\pgfplotstablecol0	\assigncontent{#1#2}
	\else\if#1w				\assigncontent{[W.]}
	\else					\assigncontent{\cite{#1#2}}
	\fi\fi
}
\frame{
	\centering\pgfplotstabletypeset[
		columns/binary/.style={column type=@{\hskip3\tabcolsep}c},
		columns/binary2/.style={column name=binary,column type=@{\hskip3\tabcolsep}c},
		columns/finite2/.style={column name=finite},
		every head row/.style={
			before row=\toprule
				&\multicolumn3c{\hskip-3\tabcolsep symmetric}
				&\multicolumn2c{\hskip-1\tabcolsep asymmetric}\\,
			after row=\midrule},
		every last row/.style={after row=\bottomrule},
		assign cell content/.code={\decodecontent####1\relax}
	]\tableRefarray
}

激®{\color{alerted text.fg}}
\frame{{Input alphabet [new idea]}
	$\begin{bmatrix}
		W(y₁|1)		&	W(y₂|1)		&	W(y₃|1)		&	⋯	\\
		W(y₁|2)		&	W(y₂|2)		&	W(y₃|2)		&	⋯	\\
		W(y₁|3)		&	W(y₂|3)		&	W(y₃|3)		&	⋯	\\
		W(y₁|4)		&	W(y₂|4)		&	W(y₃|4)		&	⋯	\\
		W(y₁|5)		&	W(y₂|5)		&	W(y₃|5)		&	⋯	\\
		W(y₁|6)		&	W(y₂|6)		&	W(y₃|6)		&	⋯	\\
		®W(y₁|6)	&	®W(y₂|6)	&	®W(y₃|6)	&	®⋯	\\
	\end{bmatrix}$
	\hfill
	\tikz[x=4em,y=-1em,baseline=-4em]{
		\draw[line width=1em+2*rule_thickness,postaction={draw=bg,line width=1em}]
			(-1,1)--(-1,7)(0,1)--(0,6)(1,1)--(1,6);
		\fill
			foreach\y in{1,...,7}{(-1,\y)circle(2pt)}
			foreach\y in{1,...,6}{(0,\y)circle(2pt)}
			foreach\y in{1,...,6}{(1,\y)circle(2pt)}
		;
		\draw[shorten <=3pt,shorten >=3pt]
			foreach\y in{1,...,6}{(-1,\y)edge[->](0,\y)}
			(-1,7)edge[->,alerted text.fg](0,6)
			foreach\y in{1,...,6}{
				(0,\y)edge[->](1,1+rnd*5)edge[->](1,1+rnd*5)edge[->](1,1+rnd*5)
			}
		;
	}
}

\frame{{Asymmetric channels \cite{HY13}}
	Recall $U_i$ is the coordinate as in $X₁^ℓ≔U₁^ℓ·G$.	\\
	The difficulty of asymmetric channels is nonuniform $U_i$.
	
	Define synthetic channel $\V k≔(U_i|U_1^{i-1})$.	\\
	Define $\V i,\VV ij,\VVV ijk,…$; define $\{𝘝_n\}$.	\\
	It polarizes, and at the same pace.
	
	High $H(𝘝_n)$ low $H(𝘞_n)$ vs both high vs both low.
}

\frame{{Bhattacharyya parameter}
	Binary $Z(W)≔∑¬_{y∈𝒴}√{W(y|0)W(y|1)}$.
	
	Non-binary
	$÷1{q-1}∑¬_{\substack{x,x'∈𝔽_q\\x≠x'}}∑¬_{y∈𝒴}√{W(x,y)W(x',y)}$. \\
	{}[New idea] $\max¬_{0≠d∈𝔽_q}∑¬_{x∈𝔽_q}∑¬_{y∈𝒴}√{W(x,y)W(x+d,y)}.$
}

\frame{{Random codes references}
	LDP: \cite{Fano61,Gallager65,SGB67,Gallager68,Gallager73,Blahut74,BF02,FLM11,DZF16}
	
	CLT: \cite{Wolfowitz57,Weiss60,Dobrushin61,Strassen62,BKB04,Hayashi09,PPV10}
	
	MDP: \cite{AW10,PV10,AW14,Arikan15p,HT15}
}

%%%% Bi-Hölder toll

\tiny
\advance\lineskip0ptplus1em
\advance\baselineskip0ptplus1em
\setbeamertemplate{bibliography item}[text]
\setbeamertemplate{bibliography entry author}{\bgroup}
\setbeamercolor{bibliography entry location}{fg=alerted text.fg}
\setbeamertemplate{bibliography entry note}{\egroup}
\bibliographystyle{plainurl}
\bibliography{Complex2Order-1}

\end{document}



